% Titlepage
$if(beamer)$
    \frame{\titlepage}
$else$

% Definerer font som skal brukes i tittel og overskrifter (bold)   

    \begin{titlepage}
        \newgeometry{top=2cm, bottom=2cm, left=2cm, right=2cm} % Adjust margins as needed

        % Background image
        \begin{tikzpicture}[remember picture, overlay]
            \node[inner sep=0pt] at (current page.center) {\includegraphics[width=\paperwidth,height=\paperheight]{$titlepage-image$}};
        \end{tikzpicture}

        % Foreground text
        \begin{tikzpicture}[remember picture, overlay]
            \node[align=center, text opacity=1, fill opacity=0] at (current page.center) {
                \begin{minipage}{\textwidth}
                    \vfill % Ensure vertical centering

                    % Title
                    \begin{flushleft}
                        \hspace*{-0.45cm} % Adjust this value to move the text to the right
                        \vspace*{-0cm} % Adjust this value to move the text upwards or downwards
                        {\roboto\textbf{\fontsize{40}{48}\selectfont $title$}}
                    \end{flushleft}

                    % Subtitle
                    \begin{flushleft}
                        \hspace*{-0.45cm} % Adjust this value to move the text to the right
                        \vspace*{5cm} % Adjust this value to move the text upwards or downwards
                        {\fontsize{14}{18}\selectfont $subtitle$}
                    \end{flushleft}

                    % Authors and Date
                    \begin{flushleft}
                        \hspace*{2cm} % Adjust this value to move the text to the right
                        \vspace*{11cm} % Adjust this value to move the text upwards or downwards
                        {\fontsize{10}{12}\selectfont \textsf{$for(author)$$author$$sep$, $endfor$}}
                        \vspace*{2cm} % Adjust this value to move the text upwards or downwards
                        {\Large \textsf{$date$}}
                    \end{flushleft}

                    \vfill
                \end{minipage}
            };
        \end{tikzpicture}

        % Legger inn rapportnummer
        $if(rapportnr)$
        \begin{textblock}{3}(-0.1 ,12.5) % Adjust the coordinates (1.4, 14) to place the text within the page
            \rotatebox{270}{\fontsize{9}{11}\selectfont $rapportnr$}
        \end{textblock}
        $endif$


    \end{titlepage}
    \restoregeometry

    $endif$

    
  
% Øyvind lager en kolofon-side
\newpage
\thispagestyle{empty}

I serien Rapporter publiseres analyser og kommenterte statistiske resultater fra ulike undersøkelser. Undersøkelser inkluderer både utvalgsundersøkelser, tellinger og registerbaserte undersøkelser.

\vskip 34em

\textcopyright
Statistisk sentralbyrå
\newline
Ved bruk av materiale fra denne publikasjonen skal Statistisk sentralbyrå oppgis som kilde.

\vskip 1em

Publisert: $publisert-dato$

\vskip 1em

ISBN 978-82-587-$isbn-t$ (trykt)
\newline
ISBN 978-82-587-$isbn-e$ (elektronisk)
\newline
ISSN 0806-2056

\vskip 0.5em
\rule{\textwidth}{0.4pt}
\vskip 0.5em
\rule{\textwidth}{0.4pt}

\noindent
\textbf{Standardtegn i tabeller} \hfill \textbf{Symbol}
\newline
\rule{\textwidth}{0.4pt}

\noindent
\textbf{Ikke mulig oppgi tall} \hfill \textbf{.}
\newline
Tall finnes ikke på dette tidspunktet fordi kategorien ikke var i bruk da tallene ble samlet inn.
\newline
\rule{\textwidth}{0.4pt}

\noindent
\textbf{Tallgrunnlag mangler} \hfill \textbf{..}
\newline
Tall er ikke kommet inn i våre databaser eller er for usikre til å publiseres.
\newline
\rule{\textwidth}{0.4pt}

\noindent
\textbf{Vises ikke av konfidensialitetshensyn} \hfill \textbf{:}
\newline
Tall publiseres ikke for å unngå å identifisere personer eller virksomheter.
\newline
\rule{\textwidth}{0.4pt}

\noindent
\textbf{Desimalskilletegn} \hfill \textbf{,}
\newline
\rule{\textwidth}{0.4pt}

\clearpage


% Defines the header and footer used in the chapters
\fancypagestyle{ssb-report-footer-header}{
    \rhead{\fontsize{9}{11}\selectfont $title$}
    \lhead{\fontsize{9}{11}\selectfont Rapporter $rapportnr$}
    \fancyfoot[C]{\fontsize{9}{11}\selectfont \thepage} % Page number in the center
    \renewcommand{\headrulewidth}{0.0pt} % Thickness of the header line
    \renewcommand{\footrulewidth}{0.0pt} % Thickness of the footer line
}


% Apply the custom page style to the entire document
\pagestyle{ssb-report-footer-header}




% Øyvind lager Forord
\newpage
\section*{Forord}
\addcontentsline{toc}{section}{Forord}
$if(preface-no)$
$preface-no$
\vskip 2em
Statistisk sentralbyrå, {$approved-date$}
\vskip 1em
$approved-by$
$else$
Skriv inn forordet her. Måten du gjør det på er legge det inn i yaml-metadatene under variabelen: forord: <teksten inn her>
$endif$
\clearpage

\newpage
\section*{Sammendrag}
\addcontentsline{toc}{section}{Sammendrag}
$if(abstract-no)$
$abstract-no$
\newpage
$endif$

\section*{Abstract}
\addcontentsline{toc}{section}{Abstract}
$if(abstract-en)$
$abstract-en$
\newpage
$endif$



